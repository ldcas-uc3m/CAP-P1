\section{Queries}


\subsection{Velocidad media de los taxis en función de la hora}
% RDDs
Para calcular la velocidad media de los taxis es necesario dividir la distancia recorrida entre el tiempo. Dado que la distancia (\textbf{\textit{trip\_distance}}) está en millas, y el tiempo (\textbf{\textit{tpep\_dropoff\_datetime}} - \textbf{\textit{tpep\_pickup\_datetime}}) en segundos, es necesario convertir a millas por segundo.

La fórmula quedaría:
\begin{equation}\label{eq:q1}
  mean\_mph = \frac{trip\_distance}{tpep\_dropoff\_datetime - tpep\_pickup\_datetime} \cdot \frac{3600\ s}{1\ h}
\end{equation}


Para realizar ésta query se ha usado el método de \textit{RDD}s. Para ello, se han seguido los siguientes pasos:
\begin{enumerate}
  \item Transformar cada entrada en una tupla con la hora de salida y la velocidad media del viaje, calculada mediante la Ecuación \ref{eq:q1} (\textit{map}).
  \item Calcular la media de las velocidades por cada hora (\textit{reduce}).
  \item Ordenar las horas en orden ascendente (\textit{sort}).
\end{enumerate}

\noindent
El resultado de la \textit{query} queda ilustrado en la \figref{q1}.

\svgfigure[.7]{q1}{Resultado de la \textit{query} 1}



\subsection{Viajes en taxi más comunes}
% RDDs / SQL / Pyspark SQL


\subsubsection*{Evaluación de la eficiencia}
\svgfigure[.7]{q2_perf}{Eficiencia de la \textit{query} 2}




\subsection{Porcentaje de propina por número de pasajeros}
% PysparkSQL


\svgfigure[.7]{q3}{Resultado de la \textit{query} 3}
