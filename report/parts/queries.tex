\section{Consultas}
Se han realizado tres \textit{queries} diferentes sobre el \textit{dataset}, mediante tres métodos distintos que proporciona \textit{Pyspark}.


\subsection{Consulta 1: Velocidad media de los taxis en función de la hora}
% RDDs
Para calcular la velocidad media de los taxis es necesario dividir la distancia recorrida entre el tiempo. Dado que la distancia (\textbf{\textit{trip\_distance}}) está en millas, y el tiempo (\textbf{\textit{tpep\_dropoff\_datetime}} - \textbf{\textit{tpep\_pickup\_datetime}}) en segundos, es necesario convertir a millas por hora.

La fórmula quedaría:
\begin{equation}\label{eq:q1}
  mean\_mph = \frac{trip\_distance}{tpep\_dropoff\_datetime - tpep\_pickup\_datetime} \cdot \frac{3600\ s}{1\ h}
\end{equation}


Para realizar esta consulta, se ha usado el método de \textit{RDD}s. Para ello, se han seguido los siguientes pasos:
\begin{enumerate}
  \item Transformar cada entrada en una tupla con la hora de salida y la velocidad media del viaje, calculada mediante la Ecuación \ref{eq:q1} (\textit{map}).
  \item Calcular la media de las velocidades por cada hora (\textit{reduce}).
  \item Ordenar las horas en orden ascendente (\textit{sort}).
\end{enumerate}

\noindent
El resultado final queda ilustrado en la \figref{q1}.

\svgfigure[.7]{q1}{Resultado de la consulta 1}



\subsection{Consulta 2: Viajes en taxi más comunes}
% SQL / Pyspark SQL / RDDs
Para identificar el viaje más común, es necesario crear grupos para cada trayecto posible, y contar el número de elementos en cada grupo.
Un trayecto, en nuestro caso, es definido cómo una ruta cualquiera desde una zona inicial (\textbf{\textit{PULocationID}}) a una zona final (\textbf{\textit{DOLocationID}}).\footnote{\textit{Se ha excluido la zona con ID 264, porque pertenece a una zona indeterminada.}}

A continuación, sin perder de vista el objetivo principal de obtener el viaje más común, se utilizan tres métodos de resolución distintos junto a sus respectivos pasos de ejecución, con el propósito de probar y comparar su velocidad de ejecución: \textit{SQL}, \textit{PySpark SQL}, y \textit{RDD}s.

\newpage
\subsubsection*{\textit{SQL}}
\begin{enumerate}
  \item Agrupar por trayectos entre dos zonas, utilizando cláusulas \textit{GROUP BY}.
  \item Ordenar en orden descendente (\textit{DESC}) en función de la cantidad de viajes.
\end{enumerate}

\subsubsection*{\textit{PySpark SQL}}
\begin{enumerate}
  \item Agrupar por trayectos entre dos zonas con la operación \textit{groupBy()}.
  \item Contar los elementos de cada grupo utilizando \textit{count()}.
  \item Ordenar en orden descendente la cantidad de viajes empleando \textit{orderBy(desc([...]))}.
\end{enumerate}

\subsubsection*{\textit{RDD}s}
\begin{enumerate}
  \item Agrupar en una tripleta las dos zonas y un contador del número de elementos (\textit{map}).
  \item Contar los elementos de cada grupo (\textit{reduceByKey()})
\end{enumerate}


\subsubsection*{Evaluación de la eficiencia}
Se observa en la \figref{q2_perf}, que el tiempo de ejecución de las consultas recogidas en este segundo apartado, es muy competitivo para \textit{SQL} y \textit{PySpark SQL}, a diferencia del método de \textit{RDD}s que tiene un peor rendimiento, incluso en ocasiones doblando el tiempo de sus adversarios.

\svgfigure[.7]{q2_perf}{Eficiencia de la consulta 2}

\newpage
\subsection{Consulta 3: Porcentaje de propina por número de pasajeros}
% PysparkSQL
Para comparar el porcentaje de propina por número de pasajeros es necesario hallar el valor de la propina \textbf{\textit{Tip\_amount}} en función del coste total del viaje \textbf{\textit{Total\_amount}}.

Para calcular el porcentaje de propina se usa la siguiente fórmula:
\begin{equation}\label{eq:q3}
  tip = \frac{Tip\_amount}{Total\_amount} \cdot 100
\end{equation}


Para elaborar esta consulta se ha empleado \textit{PySpark SQL}. Para ello, se han seguido los siguientes pasos:
\begin{enumerate}
  \item Calcular el porcentaje de propina por cada entrada, con la Ecuación \ref{eq:q3}.
  \item Agrupar los resultados por número de pasajeros utilizando \textit{groupBy()} y calcular la media con \textit{avg()}.
  \item Ordenar los resultados en orden ascendente de pasajeros con la operación \textit{sort()}.
\end{enumerate}

\noindent
El resultado final queda ilustrado en la \figref{q3}.

\svgfigure[.7]{q3}{Resultado de la consulta 3}
