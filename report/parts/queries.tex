\section{Queries}
Se han realizado tres \textit{queries} diferentes sobre el \textit{dataset}, mediante tres métodos distintos que proporciona Pyspark.


\subsection{\textit{Query} 1: Velocidad media de los taxis en función de la hora}
% RDDs
Para calcular la velocidad media de los taxis es necesario dividir la distancia recorrida entre el tiempo. Dado que la distancia (\textbf{\textit{trip\_distance}}) está en millas, y el tiempo (\textbf{\textit{tpep\_dropoff\_datetime}} - \textbf{\textit{tpep\_pickup\_datetime}}) en segundos, es necesario convertir a millas por hora.

La fórmula quedaría:
\begin{equation}\label{eq:q1}
  mean\_mph = \frac{trip\_distance}{tpep\_dropoff\_datetime - tpep\_pickup\_datetime} \cdot \frac{3600\ s}{1\ h}
\end{equation}


Para realizar ésta query se ha usado el método de \textit{RDD}s. Para ello, se han seguido los siguientes pasos:
\begin{enumerate}
  \item Transformar cada entrada en una tupla con la hora de salida y la velocidad media del viaje, calculada mediante la Ecuación \ref{eq:q1} (\textit{map}).
  \item Calcular la media de las velocidades por cada hora (\textit{reduce}).
  \item Ordenar las horas en orden ascendente (\textit{sort}).
\end{enumerate}

\noindent
El resultado de la \textit{query} queda ilustrado en la \figref{q1}.

\svgfigure[.7]{q1}{Resultado de la \textit{query} 1}



\subsection{\textit{Query} 2: Viajes en taxi más comunes}
% SQL / Pyspark SQL / RDDs
Para identificar el viaje más común es necesario crear grupos para cada trayecto posible, y contar el número de elementos en cada grupo.
Un trayecto, en nuestro caso, es definido cómo una ruta cualquiera desde una zona inicial (\textbf{\textit{PULocationID}}) a una zona final (\textbf{\textit{DOLocationID}}).

Se ha decidido excluir la zona con ID 264, ya que pertenece a una zona indeterminada.

En este caso, se prueban tres métodos distintos para comparar su velocidad de ejecución: \textit{SQL}, \textit{ PySpark SQL}, y\textit{RDD}s.


\subsubsection*{\textit{SQL}}
\begin{enumerate}
  \item Agrupar por trayectos entre dos zonas (\textit{GROUP BY}).
  \item Ordenar en orden descendente por la cantidad de viajes (\textit{DESC}).
\end{enumerate}


\subsubsection*{\textit{PySpark SQL}}
\begin{enumerate}
  \item Agrupar por trayectos entre dos zonas (\textit{groupBy}).
  \item Contar los elementos de cada grupo (\textit{count})
  \item Ordenar en orden descendente por la cantidad de viajes (\textit{DESC}).
\end{enumerate}


\subsubsection*{\textit{RDD}s}
\begin{enumerate}
  \item Agrupar en una tupla conteniendo las dos zonas y una variable que cuenta el número de elementos (\textit{map}).
  \item Contar los elementos de cada grupo (\textit{reduceByKey()})
\end{enumerate}


\subsubsection*{Evaluación de la eficiencia}
Como podemos observar en la gráfica mostrando el tiempo de ejecución para las distintas queries de la segunda pregunta, el método de \textit{RDD}s tiene una peor eficiencia comparando con \textit{SQL} o \textit{PySpark SQL}.

\svgfigure[.7]{q2_perf}{Eficiencia de las \textit{query} 2}




\subsection{\textit{Query} 3: Porcentaje de propina por número de pasajeros}
% PysparkSQL
Para comparar el porcentaje de propina por número de pasajeros es necesario hallar el valor de la propina \textbf{\textit{Tip\_amount}} con respecto al coste total del viaje \textbf{\textit{Total\_amount}}.

Para calcular el porcentaje de propina se usa la siguiente fórmula:
\begin{equation}\label{eq:q3}
  tip = \frac{Tip\_amount}{Total\_amount} \cdot 100
\end{equation}


Para realizar ésta query se ha usado el método de \textit{PySpark SQL}. Para ello, se han seguido los siguientes pasos:
\begin{enumerate}
  \item Calcular el porcentaje de propina por cada entrada, mediante la Ecuación \ref{eq:q3}.
  \item Agrupar los resultados por número de pasajeros (\textit{groupBy}) y calcular la media (\textit{avg}).
  \item Ordenar los resultados en orden ascendente de pasajeros (\textit{sort}).
\end{enumerate}


\noindent
El resultado de la \textit{query} queda ilustrado en la \figref{q3}.

\svgfigure[.7]{q3}{Resultado de la \textit{query} 3}
