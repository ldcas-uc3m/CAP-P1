\section{Preprocesado}

Para realizar correctamente el preprocesado de este laboratorio, se han realizado varias operaciones de filtrado para limpiar el \textit{dataset}, para evitar elementos incorrectos o de un valor muy elevado que sobrepase el umbral establecido. Además, antes de filtrar los datos, las columnas \textbf{\textit{tpep\_pickup\_datetime}} y \textbf{\textit{tpep\_dropoff\_datetime}} se han pasado a formato de marcas de tiempo UNIX haciendo uso de la zona horaria, en este caso, \textbf{\textit{America/New\_York}}.

Las operaciones que se han realizado en el preprocesado eliminan las filas si se encuentra algún valor nulo en las columnas \textbf{\textit{VendorID}}, \textbf{\textit{tpep\_pickup\_datetime}} y \textbf{\textit{tpep\_dropoff
\_datetime}}.

En segundo lugar, una vez limpiadas esas tres columnas, se ha obtenido información de cada columna del \textit{dataset} para tareas de filtración. Con esta información, se han eliminado las filas del conjunto de datos en las que el número de pasajeros o la distancia del viaje fuese igual a 0. También se han borrado aquellas filas en las que el tiempo de recogida de un cliente y el de llegada fueran iguales. Se fuerza además a que los valores de la columna \textbf{\textit{VendorID}} sean 1 o 2 vaciando toda fila que no cumpla este criterio. Por consiguiente, se han erradicado las filas con valor mayor o igual a 0 en la columna \textbf{\textit{extra}}.

En tercer lugar, se fuerza a que la columna \textbf{\textit{store\_and\_fwd\_flag}} tenga únicamente los valores \textbf{\textit{N}} e \textbf{\textit{Y}}, eliminando toda fila que no cumpla esto ya que únicamente puede tomar esos valores. Siguiendo esta misma justificación, se ha obligado a que las columnas \textbf{\textit{RatecodeID}} y \textbf{\textit{payment\_type}} posean un rango de valores del 1 al 6. También se han borrado todas las filas en las que \textbf{\textit{mta\_tax}} no es igual a 0.5 debido a que este valor debe de ser fijo para todas las filas. Al mismo tiempo, se han suprimido todas las filas en las que \textbf{\textit{improvement\_surcharge}} no sea mayor o igual a 0 para eliminar así valores negativos.

Adicionalmente, se han incluido filtros modificables en función de las consultas necesarias, de tal forma que se borran todas las filas en las que las columnas \textbf{\textit{fare\_amount}}, \textbf{\textit{tip\_amount}} y \textbf{\textit{tolls\_amount}} no se encuentren entre los valores 0 y 100.
