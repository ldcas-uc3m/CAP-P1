\section{Preprocesado}

Para realizar correctamente el preprocesado de este laboratorio, se han realizado varias operaciones de filtrado para limpiar el dataset de datos que fuesen incorrectos o de un valor muy elevado bajo nuestro criterio. Además, antes de filtrar los datos las columnas \textbf{\textit{tpep\_pickup\_datetime}} y \textbf{\textit{tpep\_dropoff\_datetime}} se han pasado a formato \textit{unix timestamp} haciendo uso de la zona horaria, en este caso \textbf{\textit{America/New\_York}}.

Las primeras tres operaciones que se realizan en el preprocesado son la eliminación de filas si se encuentra algún valor nulo en las columnas \textbf{\textit{VendorID}}, \textbf{\textit{tpep\_pickup\_datetime}} y \textbf{\textit{tpep\_dropoff\_datetime}}.

Una vez limpiadas esas tres columnas, se ha obtenido información de cada columna del dataset para poder así filtrar los datos. Con esta información, se han eliminado las filas del dataset en las que el número de pasajeros o la distancia del viaje fuese igual a 0. También se eliminan aquellas filas en las que el tiempo de recogida de un cliente y el tiempo de llegada fueran iguales. Se fuerza además a que los valores de la columna \textbf{\textit{VendorID}} sean 1 o 2 eliminando toda fila que no cumpla este criterio. Además, se eliminan las filas en las que el valor de la columna \textbf{\textit{extra}} no sea mayor o igual a 0.

Posteriormente, se fuerza a que la columna \textbf{\textit{store\_and\_fwd\_flag}} tenga únicamente los valores \textbf{\textit{N}} e \textbf{\textit{Y}}, eliminando toda fila que no cumpla esto. Siguiendo esta línea de filtrado, se fuerza a que las columnas \textbf{\textit{RatecodeID}} y \textbf{\textit{payment\_type}} estén en todas las filas con valores del 1 al 6. También se eliminan todas las filas en las que \textbf{\textit{mta\_tax}} no es igual a 0.5. Al mismo tiempo se eliminan todas las filas en las que \textbf{\textit{improvement\_surcharge}} no sea mayor o igual a 0.

Finalmente, los siguientes tres filtros son modificables dependiendo del criterio que se quiera seguir, en nuestro caso se ha realizado de tal forma que se eliminan todas las filas en las que las columnas \textbf{\textit{fare\_amount}}, \textbf{\textit{tip\_amount}} y \textbf{\textit{tolls\_amount}} no se encuentren entre los valores 0 y 100.
